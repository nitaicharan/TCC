% ---
% Dados Capa e Contra Capa
% ---
\author{Nitai Charan Álvares Pereira}
\title{Consultas Utilizando Linguagem Natural no Servidor de Busca Distribuído Elasticsearch}
\orientador{Fábio Falcão da França}
\local{João Pessoa - PB}
\instituicao{Centro Universitário de João Pessoa - UNIPÊ}
\data{\the\year}
\tipotrabalho{Trabalho de Concusão de Curso - TCC}
% O preambulo deve conter o tipo do trabalho, o objetivo,
	% o nome da instituição e a área de concentração
	\preambulo{Monografia apresentada ao Curso de \nomecurso do \imprimirinstituicao, como pré-requisito para a obtenção do grau de Bacharel em Ciência da Computação, sob orientação do Prof. MS.c. \imprimirorientador.}

	% ---
	% Configurações de aparência do PDF final
	% ---

	% alterando o aspecto da cor azul
	\definecolor{blue}{RGB}{41,5,195}

	% informações do PDF
	\makeatletter
	\hypersetup{
		%pagebackref=true,
			pdftitle={\@title},
			pdfauthor={\@author},
			pdfsubject={\imprimirpreambulo},
			pdfcreator={LaTeX with abnTeX2},
			pdfkeywords={abnt}{latex}{abntex}{abntex2}{trabalho acadêmico},
			colorlinks=true,                % false: boxed links; true: colored links
				linkcolor=black,                 % color of internal links
				citecolor=black,                 % color of links to bibliography
				filecolor=black,                      % color of file links
				urlcolor=black,
			bookmarksdepth=4
	}
\makeatother
% --- 

% ---
% Posiciona figuras e tabelas no topo da página quando adicionadas sozinhas
% em um página em branco. Ver https://github.com/abntex/abntex2/issues/170
\makeatletter
\setlength{\@fptop}{5pt} % Set distance from top of page to first float
\makeatother
% ---

% ---
% Possibilita criação de Quadros e Lista de quadros.
% Ver https://github.com/abntex/abntex2/issues/176
%
% configurações para atender às regras da ABNT
\setfloatadjustment{quadro}{\centering}
\counterwithout{quadro}{chapter}
\setfloatlocations{quadro}{hbtp} % Ver https://github.com/abntex/abntex2/issues/176
% ---

% --- 
% Espaçamentos entre linhas e parágrafos 
% --- 
% O tamanho do parágrafo é dado por:
\setlength{\parindent}{1.3cm}
% Controle do espaçamento entre um parágrafo e outro:
\setlength{\parskip}{0.2cm}  % tente também \onelineskip

\pdfminorversion=7
