% --- 
% Dados do TCC% 
% --- 
\newcommand{\subtitulo}{} 
\newcommand{\nomecurso}{Bacharelado em Ciência da Computação} 
\newcommand{\titulobar}{Ciência da Computação} 
\newcommand{\profa}{Nome do Professor A} 
\newcommand{\profb}{Nome do Professor B} 
\newcommand{\profc}{Nome do Professor C} 
\newcommand{\insta}{UNIPÊ} 
\newcommand{\instb}{UNIPÊ} 
\newcommand{\instc}{UNIPÊ} 
\newcommand{\coordenador}{Thatyana Carla Dias Guerra} 
\newcommand{\departamento}{Nome do Departamento} 
\newcommand{\muctbf}[1]{\MakeUppercase{\textbf{#1}}} 
\renewcommand{\ABNTEXchapterfont}{\rmfamily\bfseries} 
\newcommand{\marcador}[1]{\colorbox{yellow}{
	\begin{minipage}{\textwidth}
		\noindent{#1}
	\end{minipage}

}
}

% --- 
% Configurações do pacote backref 
% Usado sem a opção hyperpageref de backref 
\renewcommand{\backrefpagesname}{Citado na(s) página(s):~} 
% Texto padrão antes do número das páginas 
\renewcommand{\backref}{} 
% Define os textos da citação 
\renewcommand*{\backrefalt}[4]{ 
	\ifcase #1 % 
		Nenhuma citação no texto.% 
	\or 
		Citado na página #2.% 
	\else 
		Citado #1 vezes nas páginas #2.% 
	\fi}% 
	% --- 

	% --- 
	% Possibilita criação de Quadros e Lista de quadros. 
	% Ver https://github.com/abntex/abntex2/issues/176 
	% 
\newcommand{\quadroname}{Quadro} 
\newcommand{\listofquadrosname}{Lista de quadros} 

\newfloat[chapter]{quadro}{loq}{\quadroname} 
\newlistof{listofquadros}{loq}{\listofquadrosname} 
\newlistentry{quadro}{loq}{0} 

% configurações para atender às regras da ABNT 
\renewcommand{\cftquadroname}{\quadroname\space}  
\renewcommand*{\cftquadroaftersnum}{\hfill--\hfill} 
