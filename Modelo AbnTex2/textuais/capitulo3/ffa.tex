Esta sessão dedica-se a explanar mais detalhadamente todo o fluxo da aplicação proposta. Usando como base a \autoref{diagfluxo}, é descrito cada atividade contido no formato: - <Nome atividade>: Detalhamento.

\clearpage
\begin{itemize}
	\item Mostrar tela inicial - A aplicação renderiza tela inicial contendo o campo para entrada da busca e os botões de escolha Limpar e Pesquisar.
	\item Entrada de dados - É aguradado a entrada da sentença em linguagem natural solicitado pelo usuário.
	\item Selecionar opção - O usuário deve escolhe entre as opções Limpar campo e Pesquisar.
	\item Limpar - Caso o usuário escolha a opção Limpar, é limpada o campo e novamente é renderizado a tela inicial para entrada de nova busca.
	\item Pesquisar - Caso o usuário escolha a opção Pesquisar, a aplicação cria uma requisição HTTP do tipo Post contendo em seu corpo a sentença do usuário, para o endereço da API LUIS.
	\item Envia requisição ao LUIS - Após, o envio é realizado e a aplicação aguarda o retorno da solicitação.
	\item Recebe requisição - A API LUIS recebe a requisição e processa a sentença do usuário.
	\item Envia solicitação traduzida no fromato JSON - Após o processamento, a API responde a solicitação traduzindo as entidades e intenções indentificada na sentença.
	\item Recebe tradução - Recebe a tradução da sentença no formato JSON.
	\item Confirma dados recebidos - A aplicação solicita confirmação das intenção e os atores indentificado pelo LUIS estão corretas.
	\item Mostar opções Continuar e Refazer - Para confirmação, o usuário deve escolher as opções Continuar ou Refazer para dar continualidade no fluxo.
	\item Refazer - Caso escolha Refazer, é novamente mostrada a tela inicial dando a possibilidade que o usuário refaça a sentença, para posteriormente refazer novamente tradução.
	\item Continuar - Se a opção continuar for escolhida, a aplicação dará continualidade ao fluxo para busca dos dados.
	\item Envia requisição ao Elasticsearch - Posteriormente, a aplicação encaminha a requisição  de busca dos dados na base da API Elasticsearch e aguarda o retorno da requisição.
\end{itemize}
