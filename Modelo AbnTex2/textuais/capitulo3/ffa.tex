Esta sessão dedica-se a explanar mais detalhadamente todo o fluxo da aplicação proposta. Usando como base a \autoref{diagfluxo}, é descrito cada atividade contido no formato: - <Nome atividade>: Detalhamento.

\begin{enumerate}
	\item Mostrar tela inicial - A aplicação renderiza tela inicial contendo o campo para entrada da busca e os botões de escolha Limpar e Pesquisar.
	\item Entrada de dados - É aguradado a entrada da sentença em linguagem natural solicitado pelo usuário.
	\item Selecionar opção - O usuário deve escolhe entre as opções Limpar campo e Pesquisar.
	\item Limpar - Caso o usuário escolha a opção Limpar, é limpada o campo e novamente é renderizado a tela inicial para entrada de nova busca.
	\item Pesquisar - Caso o usuário escolha a opção Pesquisar, a aplicação cria uma requisição HTTP do tipo Post contendo em seu corpo a sentença do usuário, para o endereço da API LUIS.
	\item Envia requisição ao LUIS - Após, o envio é realizado e a aplicação aguarda o retorno da solicitação.
	\item Recebe requisição - A API LUIS recebe a requisição e 
\end{enumerate}
