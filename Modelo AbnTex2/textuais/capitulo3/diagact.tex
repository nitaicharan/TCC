O intuito utilizar-se do Diagrama de Atividades é dar ênfase ao fluxo de controle na execução de um comportamento realizado por um sistema, mostrando o fluxo entre atividades em um sistemas. As atividades podem ser referidas como um fluxo sequencial ou ramificação de atividades que interagem entre si e outros objetos para realizar ou sofrer ações. Descrevendo assim, uma modelagens a função do sistema \cite{Booch:2012}.  

Ao iniciar aplicação é apresentada a tela inicial contendo um campo para entrada de dados para a pesquisa e botões de escolha. Este campo é utilizado para a entrada da sentença em linguagem natural para busca do usuário que esta utilizando a aplicação. Após a inserção da sentença, é necessário a escolha da próxima ação do usuário. A ação de pesquisar e a de limpara o campo da entrada.

Ao clicar no botão de limpar, é reapresentado a tela inicial contendo novamente os botões de escolha, juntamente com o campo para a entrada para a busca. Caso o botão de pesquisar for acionado, é iniciado o processo de busca dos dados solicitado ao usuário e retornado ao mesmo caso assim os dados forem encontrados.
