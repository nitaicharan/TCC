Inteligência Artificial é uma das ramificações da ciências computacionais que se dedica ao estudo de tentar simular a inteligência humana em maquinas, desenvolver a inteligência artificial de máquinas e softwares que podem raciocinar, aprender, reunir conhecimento, comunicar, manipular e perceber objetos. Inteligência é comumente considerado como a capacidade de resolver problemas complexos realizando previamente coleta de informações e conhecimentos \cite{Pannu:2015}.

\noindent \marcador{Em um futuro próximo, máquinas inteligentes substituirão humanos com as mais diversas capacidades em várias áreas, com isso a inteligência artificial pode ser resumida em estudo e desenvolvimento de máquinas inteligentes e software que pode raciocinar aprender, reunir conhecimento, comunicar, manipular e perceber objetos.}

\subsection{Processamento de Linguagem Natural (PLN)} Processamento de Linguagem Natural (PLN) é uma subárea da Inteligência Artificial (IA) que estuda as limitações, problemas e compreensão da linguagem humana para resoluções de problemas solúveis através de uma inteligência artificial.  Seu objetivo principal é possibilitar maquinas entenderem e interpretarem a linguagem humana. Tal processo converte dados em texto em formatos que máquinas entendam como o numérico e binário \cite{Kulkarni:2019}.

Este processo é baseado em quatro etapas para extrair informações pertinentes do texto alvo. Primeiro o texto é realizado um processo que criação de \textit{tokens} das palavras do texto. Nesta etapa o texto é quebrado em unidades, sem se preocupar no significado ou classificação gramatical das palavras. Formando assim uma lista de \textit{tokens} que será usada em todos os outros processo com seus diferentes objetivos \cite{Reese:2015}.

A segunda etapa é dedicada a detecção de sentenças utilizando unidades criados na primeira etapa. A combinação de palavras em uma frase ou uma sentença pode ter diferentes significados mudando o relacionamento entre outras palavra ou outras sentenças. Pode ser dado como exemplo que, frequente é distinguido a função gramatical entre as palavras como substantivo e verbos \cite{Reese:2015}.

A terceira é o processo de classificada cada palavra e documentos que consiste na rotulação destas informações encontradas no texto. No processo de rotulação, o rótulo pode ou não existir antes desta etapa. Caso exista, esta etapa é chamada de classificação. Antagônico, é chamado de agrupamento \cite{Reese:2015}.

A quarta etapa é o processo de extração de relacionamentos identificado no texto.

Relationship extraction

relationships exist in text

humans can determining how things are related to each other
