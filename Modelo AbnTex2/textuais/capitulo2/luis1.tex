	\begin{itemize}
		\item creating a new LUIS “application”
		\item specifying the intents and entities
		\item enter a few utterances
		\item intent label
			\begin{itemize}
				\item choosing from a drop-down
				\item specify any entities
			\end{itemize}

		\item enters labels, the model is automatically and asynchronously re-built
		\item the current model is used to propose labels when new utterances are entered
		\item These proposed labels serve two purposes
			\begin{itemize}
				\item they act as a rotating test set and illustrate the performance of the current model on unseen data
				\item when the proposed labels are correct, they act as an accelerator.
			\end{itemize}
		\item shows several visualizations
		\item error surfaces in a visualization options for fixing
			\begin{itemize}
				\item add more labels
				\item they can change a la bel
				\item add a feature
					\begin{itemize}
						\item is a dictionary of words or phrases
						\item will be used by the machine learning algorithm
						\item particularly useful for helping the models to generalize
					\end{itemize}
			\end{itemize}
		\item add “pre-built” ready-to-use
			\begin{itemize}
				\item entities
				\item including numbers
				\item temperatures
				\item locations
				\item monetary amounts
				\item ages
				\item encyclopaedic concepts,
				\item dates
				\item and times
			\end{itemize}
		\item “publish” models to an HTTP endpoint point takes the utterance text as input
		\item returns an object in JavaScript Object Notation (JSON) form.
		\item The endpoint is accessible by any internet-connected device, including mobile phones, tablets, wearables, robots, and embedded devices; and embedded devices; and is optimized for real-time operation.
		\item As utterances are received on the HTTP end-point, they are logged and are available for labeling in LUIS. 
		\item receive substantial usage, so labeling every utterance would be inefficient. LUIS provides two
			ways of managing large scale traffic efficiently.
				First, a conventional (text) search index is created
				which allows a developer to search for utterances
				that contain a word or phrase, like “switch on” or
				“air conditioning”. This lets a developer explore
				the data to look for new intents or entirely new
				phrasings. Second, LUIS can suggest the most
				useful utterances to label by using active learning.
				Here, all logged utterances are scored with the cur-
				rent model, and utterances closest to the decision
				boundary are presented first. This ensures that the
				developer’s labeling effort has maximal impact.
	\end{itemize}
