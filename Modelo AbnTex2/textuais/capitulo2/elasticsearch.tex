Para busca e persistência dos dados será utilizado a tecnologia Elasticsearch que se caracteriza por ser um mecanismo de busca open-source que realiza buscas e analisar dados em tempo real.

Tendo como principal objetivo facilitar a busca por texto e a rapidez no acesso aos dados e gravação, o Elasticsearch possui a capacidade de funcionamento escalável sendo possível a utilização em apenas um servidor (\textit{standalone}) ou de forma distribuída em centenas de servidores.  

Para busca é disponibilizada uma interface RESTful API que possibilita a obtenção e visualização dos dados por aplicações \textit{web client} \cite{Gormley:2015}.

Possuindo como base o Apach Lucene, qualifica por ser atualmente a mais avançado, performática e completa livrara de motor de busca existente, é construído na linguagem Java para poder integrar e usufruir dessa livraria. Ao contrário, é necessário uma grande carga de conhecimento necessária para compreender como é realizado as facilidades tragas pelo Lucene devido a sua complexidade \cite{Gormley:2015}.

Um dos maiores benefícios tragos através da utilização do Apach Lucene é a ação de indexação. Utilizando disso, o Elasticsearch realiza busca, ordena, e filtra persisti dados fornecidos como objeto e documentos realizando a indexação em todas suas ações. Consequentemente, esta tecnologia também é caracterizado como orientado a documento.

Um documento é um dado possuinte de campos como linhas e colunas
Consequentemente, esta tecnologia também é caracterizado como orientado a documento.

Como forma de intervenção a esta complexidade, o Elasticsearch foi criado para  ser capas de, utilizando os benefícios herdado pela utilização do Lucene, disponibiliza dados solicitados através de uma simples uma API RESTful. Porém, esta tecnologia é mais do que somente o Lucene, um motor de busca por texto, é um um conjunto de ferramentas?????

busca realizar a persistência dos dados de forma distribuída no formato de documentos os quais indexa todos os campos para que posteriormente seja possível a localização de forma rápida \cite{Gormley:2015}. 
