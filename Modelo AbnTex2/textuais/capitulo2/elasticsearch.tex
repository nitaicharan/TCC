Para busca e persistência dos dados será utilizada a tecnologia \textit{Elasticsearch}, que se caracteriza por ser um conjunto de mecanismo de busca \textit{open-source} que realiza buscas e analisa dados em tempo real \cite{Gil:2010}. Através disso, são criada soluções próprias que utilizam as  facilidades trazidas pelo conjunto para atingir seus objetivos.

Tendo como principal objetivo facilitar a busca por texto e a rapidez no acesso aos dados e gravação, o \textit{Elasticsearch} possui a capacidade de funcionamento escalável, sendo possível a utilização em apenas um servidor (\textit{standalone}) ou de forma distribuída em centenas de servidores \cite{Gormley:2015}, agilizando assim a entrega ao usuário ou aplicação que solicitar os dados persistidos.

Para solicitar os dados persistidos é disponibilizada uma interface RESTful API que possibilita a obtenção e visualização dos dados por aplicações \textit{web client} \cite{Gormley:2015}. Esta funcionalidade disponibiliza a comunicação entre o \textit{Elasticsearch} e outras aplicações independentemente de como foram feitas ou da linguagem utilizada na construção das aplicações \textit{web clients}. Gera-se assim, independência e maior usabilidade.

Possuindo como base o \textit{Apach Lucene}, o qual se qualifica por ser atualmente a livraria de motor de busca mais avançada, performática e completa  existente, o Elasticsearch é construído na linguagem Java para poder integrar e usufruir dessa livraria. Ao contrário, é necessário um grande volume de conhecimento para compreender como são realizada as facilidades trazidas pelo \textit{Lucene} devido a sua complexidade \cite{Gormley:2015}.

Um dos maiores benefícios trazidos pela da utilização do \textit{Apach Lucene} é a ação de indexação. Utilizando isso, o \textit{Elasticsearch} realiza busca, ordena, filtra e persiste dados fornecidos como objetos e documentos realizando a indexação em todas as suas ações. Consequentemente, esta tecnologia também é caracterizada como orientada a documento \cite{Gormley:2015}.

Um documento é um dado constituído por campos o qual pode se repetir por várias vezes. Todos os campos possui em tipo como texto, numérico, data, etc... Ou tipos mais complexos como subdocumentos ou \textit{arrays}. Para cada documento persistido é salvo um pequeno título, data de publicação e um \textit{link} de acesso ao conteúdo do persistido. O ato de se persistir um documento é chamado de indexação e o nome do dado persistido é chamado de índice \cite{Kuc:2013}.

Após a criação desta estrutura, é possível realizar pesquisas através de temas. Agilizando assim, a obtenção dos dados desejados. A disponibilização dos dados solicitados ocorre através de uma simples API RESTful mediante a transformação e transporte dos dados no formato \textit{JavaScript Object Notation}(JSON) como exemplificado na \autoref{pretty}.
