Para busca e persistência dos dados será utilizado a tecnologia Elasticsearch que se caracteriza por ser um mecanismo de busca open-source que realiza buscas e analisar dados em tempo real.

Tendo como principal objetivo facilitar a busca por texto e a rapidez no acesso aos dados e gravação, o Elasticsearch possui a capacidade de funcionamento escalável sendo possível a utilização em apenas um servidor (\textit{standalone}) ou de forma distribuída em centenas de servidores.  Para busca é disponibilizada uma interface RESTful API que possibilita a obtenção e visualização dos dados por aplicações \textit{web client} \cite{Gormley:2015}.

Possuindo como base o Apach Lucene, qualifica por ser atualmente a mais avançado, performático e completa livrara de motor de busca existente,

busca realizar a persistência dos dados de forma distribuída no formato de documentos os quais indexa todos os campos para que posteriormente seja possível a localização de forma rápida \cite{Gormley:2015}. 
