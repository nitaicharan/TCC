Para busca e persistência dos dados será utilizado a tecnologia Elasticsearch que se caracteriza por ser um conjunto de mecanismo de busca open-source que realiza buscas e analisar dados em tempo real \cite{Gil:2010}. Estas características são obtidas devido ao uso do conjunto de tecnologia agregado que o Elasticsearch possui. Através disso é criada soluções próprias que utilizam as  facilidades tragas do conjunto para atingir seus objetivos.

Tendo como principal objetivo facilitar a busca por texto e a rapidez no acesso aos dados e gravação, o Elasticsearch possui a capacidade de funcionamento escalável sendo possível a utilização em apenas um servidor (\textit{standalone}) ou de forma distribuída em centenas de servidores \cite{Gormley:2015}. Agilizando assim, a entrega ao usuário ou aplicação que solicitar os dados persistidos.

Para solicitar os dados persistido é disponibilizada uma interface RESTful API que possibilita a obtenção e visualização dos dados por aplicações \textit{web client} \cite{Gormley:2015}. Esta funcionalidade disponibiliza a comunicação entre o Elasticsearch e outras aplicações independente como foi feito ou a linguagem da aplicação \textit{web client}. Gerando assim, independência e maior usabilidade.

Possuindo como base o Apach Lucene, qualifica por ser atualmente a mais avançado, performática e completa livrara de motor de busca existente, é construído na linguagem Java para poder integrar e usufruir dessa livraria. Ao contrário, é necessário uma grande carga de conhecimento necessária para compreender como é realizado as facilidades tragas pelo Lucene devido a sua complexidade \cite{Gormley:2015}.

Um dos maiores benefícios tragos através da utilização do Apach Lucene é a ação de indexação. Utilizando disso, o Elasticsearch realiza busca, ordena, e filtra persisti dados fornecidos como objeto e documentos realizando a indexação em todas suas ações. Consequentemente, esta tecnologia também é caracterizado como orientado a documento \cite{Gormley:2015}.

Um documento é um dado constituído por campos, o qual pode se repetir por várias vezes. Todos os campos possui em tipo como texto, numérico, data, etc... Ou tipos mais complexos como subdocumentos ou \textit{arrays} \cite{Kuc:2013}. Para cada documento persistido é salvo um pequeno título, data de publicação e um link de acesso ao conteúdo do persistido. O ato de se persistir um documento é chamado de indexação e o nome do dado persistido é chamado de índice. 

Após a criação desta estrutura, é possível realizar buscas através de temas otimizando assim,  o tempo necessário para a análise dos dados. Esta estrutura  

Como forma de intervenção a esta complexidade, o Elasticsearch foi criado para  ser capas de, utilizando os benefícios herdado pela utilização do Lucene, disponibiliza dados solicitados através de uma simples API RESTful. A troca de informações entre as aplicações é realizada em \textit{JavaScript Object Notation}(JSON)
