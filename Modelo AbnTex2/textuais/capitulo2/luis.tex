Language Understanding Intelligent Service (LUIS) é uma suite de \textit{software} pertencente a um dos Serviços Cognitivos da Microsoft que utiliza um ramo da Inteligência Artificial (IA) chamado Processamento de Linguagem Natural (PLN) \cite{Mayo:2017}. Essa suite, possibilita que aplicações possam adquirir certo grau de inteligência. Formando assim, seu objetivo central da utilização.

O objetivo de utilização desta ferramenta é possibilitar a tradução de linguagem natural em texto plano para que programas de computadores possam entender e interagir utilizando tais dados. Para isso, a API extrai de uma sentença do tipo texto em linguagem natural as intenções e entidades, retornando após objetos no formato JavaScript Object Notation (JSON) \cite{Mayo:2017}.

Previamente, é necessário a criação de modelos de linguagens para extração das entidades e intenções. Qualquer objeto, seja ele abstrato ou concreto, podem ser caracterizados como entidades. Exemplos como lugares, tempo e números podem ser definidos como entidades alvo. Consequentemente possibilitando o reconhecimento de comandos dos usuários em entradas como frases ou sentenças em linguagem natural \cite{Larsen:2017}.

Porém, a LUIS não é uma aplicação cliente que deve conversar e solicitar dados ao cliente. LUIS possibilita para que outras aplicações como \textit{chatbots}, \textit{websites}, aplicativos móveis ou aplicações \textit{descktops} possão se comunicar via mediante a \textit{internet} via API.
