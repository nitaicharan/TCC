Elasticsearch is an open-source search engine built on top of Apache Lucene™, a full-
text search-engine library. Lucene is arguably the most advanced, high-performance,
and fully featured search engine library in existence today—both open source and
proprietary.
But Lucene is just a library. To leverage its power, you need to work in Java and to
integrate Lucene directly with your application. Worse, you will likely require a
degree in information retrieval to understand how it works. Lucene is very complex.
Elasticsearch is also written in Java and uses Lucene internally for all of its indexing
and searching, but it aims to make full-text search easy by hiding the complexities of
Lucene behind a simple, coherent, RESTful API.
However, Elasticsearch is much more than just Lucene and much more than “just”

Elasticsearch is an open-source search engine built on top of Apache Lucene™, a full-
text search-engine library. Lucene is arguably the most advanced, high-performance,
and fully featured search engine library in existence today—both open source and
proprietary.
But Lucene is just a library. To leverage its power, you need to work in Java and to
integrate Lucene directly with your application. Worse, you will likely require a
degree in information retrieval to understand how it works. Lucene is very complex.
Elasticsearch is also written in Java and uses Lucene internally for all of its indexing
and searching, but it aims to make full-text search easy by hiding the complexities of
Lucene behind a simple, coherent, RESTful API.
However, Elasticsearch is much more than just Lucene and much more than “just”
full-text search. It can also be described as follows:
\begin{itemize}
	\item Nitai
\end{itemize}
