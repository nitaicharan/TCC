
Continuing our new searchg aplication, let's assume we fetched some news bits from a custom seorce. the following shows the two news items that we are going to add to our indexa

For each news bit, we have a title, publishing date, content, and link, which are the constituents of the typical infromation in a news article. We will treat each news item as a document and add it to our news data store. The act fo adding documents to the data store is called indexing and the data store itself is called an index. Once the index is created, you can query it to locate documents by search terms, and this is what's referred to as searching th index

uses Lucene internally for all of its indexing and searching

aims to make full-text search easy by hiding the complexities of Lucene behind a simple, coherent, RESTful API

However, Elasticsearch is much more than just Lucene and much more than “just” full-text search

Marvel is a management and monitoring tool for Elasticsearch

, which is free for development use

It comes with an interactive console called Sense

, which makes it easy to talk to Elasticsearch directly from your browser

Marvel is available as a plugin

To download and install it, run this command in the Elasticsearch directory: ./bin/plugin -i elasticsearch/marvel/latest 

Viewing Marvel and Sense If you installed the Marvel management and monitoring tool, 

you can view it in a web browser by visiting \url{http://localhost:9200/_plugin/marvel/}

You can reach the Sense developer console either by clicking the “Marvel dashboards” drop-down in Marvel, or by visiting \url{http://localhost:9200/_plugin/marvel/sense/}
