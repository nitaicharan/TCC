Após esse capítulo introdutório, o conteúdo deste trabalho organiza-se da seguinte forma:
\begin{itemize}
	\item Capítulo 2 - \uppercase{Conjuntos de tecnologias utilizadas}: apresentará o conjunto de ferramentas que vão possibilitar a criação da ferramenta proposta.
	\item Capítulo 3 - \uppercase{Arquitetura da solução}: apresentará a arquitetura da solução proposta, detalhando seu fluxo de funcionamento e diagrama para melhor compreensão da aplicação.
	\item Capítulo 4 - \uppercase{Preparo do ambiente}: apresentará como ocorreu a réplica da base de dados governamental para buscas dos usuários e como foram gerado os algoritmos necessários para utilização do LUIS.
	\item Capítulo 5 - \uppercase{Aplicação proposta}: apresentará a aplicação, juntamente com seus diagramas para melhor entendimento da arquitetura utilizada para criação.
	\item Capítulo 6 - \uppercase{Conclusão}: finaliza o trabalho apresentando os resultados obtidos, formas de utilizar o modelo para entender o desempenho dos alunos, discutindo os possíveis trabalhos futuros.
\end{itemize}
