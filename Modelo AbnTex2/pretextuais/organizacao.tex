Após esse capítulo introdutório, o conteúdo deste trabalho organiza-se da seguinte forma:
\marcador{
	\begin{itemize}
		\item{Capítulo <n> - \uppercase{DESENVOLVIMENTO DA SOLUÇÃO}: apresentará a ferramenta proposta, assim como arquitetura de software, diagramas, mockups de telas e demais artefatos que contribuíram com o trabalho;}
		\item{Capítulo <n> - \uppercase{<Título>}: apresentará quais teorias e respectivos autores mais contribuíram para a realização do estudo e as bases teóricas para a realização deste trabalho;}
		\item{Capítulo <n> - \uppercase{<Título>}: apresentará os conceitos e as pesquisas que fundamentam a realização deste trabalho;}
		\item{Capítulo 2 - \uppercase{Estudo a respeito da API Elasticsearch}: apresentará um estudo detalhado a respeito da ferramenta utilizada para buscar, gravar e extrar informações de forma rápida e indexada;}
		\item{Capítulo 3 - \uppercase{Estudo a respeito da Language Understanding Intelligent Service (LUIS)}: apresentará um estudo detalhado a respeito da ferramenta de reconhecimento textual;}
		\item{Capítulo 4 - CONCLUSÃO: finaliza o trabalho apresentando os resultados obtidos, formas de utilizar o modelo para entender o desempenho dos alunos, discutindo os possíveis trabalhos futuros.}
	\end{itemize}
}
