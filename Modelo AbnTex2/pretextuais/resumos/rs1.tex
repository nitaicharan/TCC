Políticas públicas de transparência e fiscalização são essenciais para que o cidadão disponha de controle na fiscalização de gastos públicos. Porém, tais cidadães certamente podem utilizar a \textit{internet} como canal mediador entre as entidade portadora dos dados para fiscalização pública e o cidadão que atua como fiscal neste cenário. A \textit{internet} acaba excluindo, assim, pessoas leigas em elementos com teor tecnológico.

Em vista disso, este trabalho propõe a criação de uma aplicação que auxilie na busca dos dados dados públicos persistidos em uma base de dados. Para facilitar a utilização, será utilizada a linguagem natural possibilitando a pessoas leigas em assuntos de conhecimentos tecnológicos, possam utilizar a aplicação e poder ajudar na fiscalização dos gastos governamentais.

Para o processamento da linguagem natural, será utilizado a suite de \textit{software} Language Understanding Intelligent Service (LUIS) pertencente a um dos Serviços Cognitivos da Microsoft que utiliza um ramo da Inteligência Artificial (IA) chamado Processamento de Linguagem Natural (PLN).
