Políticas públicas de transparência e fiscalização são essenciais para que o cidadão disponha de controle na fiscalização de gastos públicos. Porém, tais cidadães certamente podem utilizar a \textit{internet} como canal mediador entre as entidade portadora dos dados para fiscalização pública e o cidadão que atua como fiscal neste cenário. A \textit{internet} acaba excluindo, assim, pessoas leigas em elementos com teor tecnológico.

porém essas ferramentas apesar de apresentarem soluções eficazes muitas vezes não possuem um processo de inclusão realmente eficiente e condizente com a realidade do usuário final. 

Muitas dar aplicações existentes direcionadas ao público de portadores do espectro autista possuem como proposito auxiliar na comunicação e na aprendizagem da língua portuguesa ou de atividades do cotidiano, satisfazendo apenas campo específicos e havendo a real necessidade de mecanismos que auxiliem em outras áreas.

Portanto o presente trabalho tem o intuito de gerar uma aplicação de apoio ao ensino dos numerais de 1 a 9 para crianças portadoras do transtorno do espectro autista, utilizando um processo criativo baseado no usuário

, o Design Thinking, cujo objetivo principal é a resolução de problemas com foco primordial no ser humano, levando em consideração a realidade do público alvo, satisfazendo as suas necessidades e respeitando as suas limitações.  

Através de ciclo de etapas baseados nesta metodologia foi possível realizar o desenvolvimento da solução proposta, cuja última etapa consiste na validação da solução, através da aplicação da mesma em campo por crianças portadoras do espectro.
