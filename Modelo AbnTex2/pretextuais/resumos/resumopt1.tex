\begin{enumerate}
\item Tecnologia no problema
\item Apresentar limitação das aplicações existentes
\item Mostrar que não há aplicações direcionadas a este propósito
\item Mostrar aplicação (solução) que ajudará no problema apresentado
\item Falar sobre a metodologia
\item Falar sobre a o desenvolvimento da aplicação seguindo a metodologia
\end{enumerate}


\newpage
{\fontsize{17}{0}\selectfont Construção do Resumo}
\begin{itemize}
\item Indicativo sobre o trabalho e não sobre o conteúdo do trabalho
\item Entre 150 e 500 palavras
\item Não usar abreviações, símbolos, formulas, equações ou diagramas a não ser que seje estritamente indispensáveis
\item Não usar citações
\item Não faça mais de um paragrafo
\item Não crie tópicos
\item Sem frases negativas
\item Após o título do trabalho e nome dos autores
\item Primeira frase indica o tema do trabalho
\item Informar os objetivos geral e específicos, metodologia, resultados 
\item 3 a 5 palavras chaves que represente a essência do trabalho separadas por pontos
\end{itemize}

{\fontsize{17}{0}\selectfont Partes do Resumo}
\begin{itemize}
\item Introdução

Objetivos e relevância do trabalho do trabalho. 
\item Metodologia

Elementos essenciais do método de pesquisa
\item Resultados

Principais resultados ou de maior destaque que pode ser representativo da pesquisa
\item Discussão
\item Conclusão

Contribuição e limites do trabalho
\end{itemize}

{\fontsize{17}{0}\selectfont Normas ABNT resumo}
\begin{itemize}
\item Frases afirmativas, curtas, escritas em voz ativa e na terceira pessoa do singular
\item Texto claro, conciso, seguir uma ordem lógica e sem completamente fiel ao trabalho
\item Deve mencionar o tema, objetivo, métodos, resultados e conclusões
\item Deve indicar, no fim, palavras chaves
\end{itemize}
