A partir da promulgação da Lei de Acesso à Informação, Lei nº 12.527/2011, foi garantido o direito constitucional de acesso às informações públicas, possibilitando que qualquer pessoa física ou jurídica, sem a necessidade de apresentar motivos, receba informações públicas de órgãos e entidades. Desta forma, surgem novos mecanismos os quais possibilitam a participação do cidadão na fiscalização de gastos públicos e combate à corrupção.

		A melhoria do acesso à informação pública e a criação de regras que permitem a disseminação de informações produzidas pelo governo reduzem os abusos que podem ser cometidos \cite{Islam:2002}. Porém, tais dados estão mais disponíveis em sistemas computacionais mediante a internet, utilizando-se muitas vezes de ferramentas carentes de parametrização ou disponibilização dos dados em grandes planilhas, dificultando assim o entendimento por parte de não-especialistas em informática.

		Na tentativa de melhorar a divulgação dos dados públicos, o Ministério da Transparência em conjunto com a Controladoria-Geral da União, em 2004, criou o site Portal da Transparência do Governo Federal que possibilita o acesso livre, no qual o cidadão pode encontrar informações sobre como o dinheiro público é utilizado, além de se informar sobre assuntos relacionados à gestão pública do Brasil.  Ainda assim, o problema de acesso por pessoas sem muitos conhecimentos técnicos sobre internet e como utilizá-la continua a existir.

		Como forma de intervenção, este trabalho sugere a utilização de linguagem natural para busca e manipulação dos dados em bases de dados disponibilizadas pelo Governo Federal. “O processamento de linguagem natural permite ao computador compreender e reagir a declarações e comandos de voz realizados em uma linguagem natural” \cite[p. 508]{Stair:2015}. Para tal, este trabalho propõe a utilização do serviço de API de reconhecimento vocal Language Understanding Intelligent Service (LUIS) que aplica inteligência de aprendizado personalizado de máquina a um texto de linguagem natural e extrai informações relevantes a futuras aplicações.  Para isso, este trabalho também propõe a utilização da ferramenta Elasticsearch que, entre muitas funcionalidades, possui recursos e tecnologias que permitem realizar consultas através de índices em grandes volumes de dados em tempo real \cite{Gil:2010}.
