% resumo em português
\setlength{\absparsep}{18pt} % ajusta o espaçamento dos parágrafos do resumo
\begin{resumo}
	O transtorno do espectro autista consiste em uma condição neurológica permanente que se manifesta na infância comprometendo a capacidade de comunicação, interação social, aprendizagem e adaptação ao meio do indivíduo portador. 

	O uso de tecnologias assistivas na área educacional tem potencializado o ensino e o desenvolvimento de portadores do espectro, auxiliando no processo de inclusão escolar e na adaptação dos modos de aquisição de conhecimentos, 

	porém essas ferramentas apesar de apresentarem soluções eficazes muitas vezes não possuem um processo de inclusão realmente eficiente e condizente com a realidade do usuário final. 

	Muitas dar aplicações existentes direcionadas ao público de portadores do espectro autista possuem como proposito auxiliar na comunicação e na aprendizagem da língua portuguesa ou de atividades do cotidiano, satisfazendo apenas campo específicos e havendo a real necessidade de mecanismos que auxiliem em outras áreas.

	Portanto o presente trabalho tem o intuito de gerar uma aplicação de apoio ao ensino dos numerais de 1 a 9 para crianças portadoras do transtorno do espectro autista, utilizando um processo criativo baseado no usuário

	, o Design Thinking, cujo objetivo principal é a resolução de problemas com foco primordial no ser humano, levando em consideração a realidade do público alvo, satisfazendo as suas necessidades e respeitando as suas limitações.  

	Através de ciclo de etapas baseados nesta metodologia foi possível realizar o desenvolvimento da solução proposta, cuja última etapa consiste na validação da solução, através da aplicação da mesma em campo por crianças portadoras do espectro.

	\textbf{Palavras-chave}: latex. abntex. editoração de texto.
	\newpage
	\begin{enumerate}
		\item Apresentar problema
		\item Tecnologia ajuda espectro auxilando na inclusão nas escolas
		\item apresentar problemas das ferramentas
		\item ferramentas apenas auxiliam em outras áreas
		\item Falar sobre a aplicação de apoio
		\item Falar sobre Design Thinking
		\item Falar sobre o desenvolvimento através das etapas propostas pela metolologia
	\end{enumerate}


	\newpage
	{\fontsize{17}{0}\selectfont Construção do Resumo}
	\begin{itemize}
		\item Indicativo sobre o trabalho e não sobre o conteúdo do trabalho
		\item Entre 150 e 500 palavras
		\item Não usar abreviações, símbolos, formulas, equações ou diagramas a não ser que seje estritamente indispensáveis
		\item Não usar citações
		\item Não faça mais de um paragrafo
		\item Não crie tópicos
		\item Sem frases negativas
		\item Após o título do trabalho e nome dos autores
		\item Primeira frase indica o tema do trabalho
		\item Informar os objetivos geral e específicos, metodologia, resultados 
		\item 3 a 5 palavras chaves que represente a essência do trabalho separadas por pontos
	\end{itemize}

	{\fontsize{17}{0}\selectfont Partes do Resumo}
	\begin{itemize}
		\item Introdução

			Objetivos e relevância do trabalho do trabalho. 
		\item Metodologia

			Elementos essenciais do método de pesquisa
		\item Resultados

			Principais resultados ou de maior destaque que pode ser representativo da pesquisa
		\item Discussão
		\item Conclusão

			Contribuição e limites do trabalho
	\end{itemize}

{\fontsize{17}{0}\selectfont Normas ABNT resumo}
\begin{itemize}
	\item Frases afirmativas, curtas, escritas em voz ativa e na terceira pessoa do singular
	\item Texto claro, conciso, seguir uma ordem lógica e sem completamente fiel ao trabalho
	\item Deve mencionar o tema, objetivo, métodos, resultados e conclusões
	\item Deve indicar, no fim, palavras chaves
\end{itemize}

\end{resumo}

% resumo em inglês
\begin{resumo}[Abstract]
	\begin{otherlanguage*}{english}
		This is the english abstract.

		\vspace{\onelineskip}

		\noindent 
		\textbf{Keywords}: latex. abntex. text editoration.
	\end{otherlanguage*}
\end{resumo}

% resumo em francês 
%\begin{resumo}[Résumé]
% \begin{otherlanguage*}{french}
%    Il s'agit d'un résumé en français.
% 
%   \textbf{Mots-clés}: latex. abntex. publication de textes.
% \end{otherlanguage*}
%\end{resumo}
%
% resumo em espanhol
%\begin{resumo}[Resumen]
% \begin{otherlanguage*}{spanish}
%   Este es el resumen en español.
%  
%   \textbf{Palabras clave}: latex. abntex. publicación de textos.
% \end{otherlanguage*}
%\end{resumo}
