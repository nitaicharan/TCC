\batchmode


\documentclass[12pt,openright,oneside,a4paper,english,french,spanish,brazil]{elementos/customizacao}


\makeindex




\usepackage{xcolor}

\usepackage[latin1]{inputenc}



\makeatletter

\makeatletter
\count@=\the\catcode`\_ \catcode`\_=8 
\newenvironment{tex2html_wrap}{}{}%
\catcode`\<=12\catcode`\_=\count@
\newcommand{\providedcommand}[1]{\expandafter\providecommand\csname #1\endcsname}%
\newcommand{\renewedcommand}[1]{\expandafter\providecommand\csname #1\endcsname{}%
  \expandafter\renewcommand\csname #1\endcsname}%
\newcommand{\newedenvironment}[1]{\newenvironment{#1}{}{}\renewenvironment{#1}}%
\let\newedcommand\renewedcommand
\let\renewedenvironment\newedenvironment
\makeatother
\let\mathon=$
\let\mathoff=$
\ifx\AtBeginDocument\undefined \newcommand{\AtBeginDocument}[1]{}\fi
\newbox\sizebox
\setlength{\hoffset}{0pt}\setlength{\voffset}{0pt}
\addtolength{\textheight}{\footskip}\setlength{\footskip}{0pt}
\addtolength{\textheight}{\topmargin}\setlength{\topmargin}{0pt}
\addtolength{\textheight}{\headheight}\setlength{\headheight}{0pt}
\addtolength{\textheight}{\headsep}\setlength{\headsep}{0pt}
\setlength{\textwidth}{349pt}
\newwrite\lthtmlwrite
\makeatletter
\let\realnormalsize=\normalsize
\global\topskip=2sp
\def\preveqno{}\let\real@float=\@float \let\realend@float=\end@float
\def\@float{\let\@savefreelist\@freelist\real@float}
\def\liih@math{\ifmmode$\else\bad@math\fi}
\def\end@float{\realend@float\global\let\@freelist\@savefreelist}
\let\real@dbflt=\@dbflt \let\end@dblfloat=\end@float
\let\@largefloatcheck=\relax
\let\if@boxedmulticols=\iftrue
\def\@dbflt{\let\@savefreelist\@freelist\real@dbflt}
\def\adjustnormalsize{\def\normalsize{\mathsurround=0pt \realnormalsize
 \parindent=0pt\abovedisplayskip=0pt\belowdisplayskip=0pt}%
 \def\phantompar{\csname par\endcsname}\normalsize}%
\def\lthtmltypeout#1{{\let\protect\string \immediate\write\lthtmlwrite{#1}}}%
\newcommand\lthtmlhboxmathA{\adjustnormalsize\setbox\sizebox=\hbox\bgroup\kern.05em }%
\newcommand\lthtmlhboxmathB{\adjustnormalsize\setbox\sizebox=\hbox to\hsize\bgroup\hfill }%
\newcommand\lthtmlvboxmathA{\adjustnormalsize\setbox\sizebox=\vbox\bgroup %
 \let\ifinner=\iffalse \let\)\liih@math }%
\newcommand\lthtmlboxmathZ{\@next\next\@currlist{}{\def\next{\voidb@x}}%
 \expandafter\box\next\egroup}%
\newcommand\lthtmlmathtype[1]{\gdef\lthtmlmathenv{#1}}%
\newcommand\lthtmllogmath{\dimen0\ht\sizebox \advance\dimen0\dp\sizebox
  \ifdim\dimen0>.95\vsize
   \lthtmltypeout{%
*** image for \lthtmlmathenv\space is too tall at \the\dimen0, reducing to .95 vsize ***}%
   \ht\sizebox.95\vsize \dp\sizebox\z@ \fi
  \lthtmltypeout{l2hSize %
:\lthtmlmathenv:\the\ht\sizebox::\the\dp\sizebox::\the\wd\sizebox.\preveqno}}%
\newcommand\lthtmlfigureA[1]{\let\@savefreelist\@freelist
       \lthtmlmathtype{#1}\lthtmlvboxmathA}%
\newcommand\lthtmlpictureA{\bgroup\catcode`\_=8 \lthtmlpictureB}%
\newcommand\lthtmlpictureB[1]{\lthtmlmathtype{#1}\egroup
       \let\@savefreelist\@freelist \lthtmlhboxmathB}%
\newcommand\lthtmlpictureZ[1]{\hfill\lthtmlfigureZ}%
\newcommand\lthtmlfigureZ{\lthtmlboxmathZ\lthtmllogmath\copy\sizebox
       \global\let\@freelist\@savefreelist}%
\newcommand\lthtmldisplayA{\bgroup\catcode`\_=8 \lthtmldisplayAi}%
\newcommand\lthtmldisplayAi[1]{\lthtmlmathtype{#1}\egroup\lthtmlvboxmathA}%
\newcommand\lthtmldisplayB[1]{\edef\preveqno{(\theequation)}%
  \lthtmldisplayA{#1}\let\@eqnnum\relax}%
\newcommand\lthtmldisplayZ{\lthtmlboxmathZ\lthtmllogmath\lthtmlsetmath}%
\newcommand\lthtmlinlinemathA{\bgroup\catcode`\_=8 \lthtmlinlinemathB}
\newcommand\lthtmlinlinemathB[1]{\lthtmlmathtype{#1}\egroup\lthtmlhboxmathA
  \vrule height1.5ex width0pt }%
\newcommand\lthtmlinlineA{\bgroup\catcode`\_=8 \lthtmlinlineB}%
\newcommand\lthtmlinlineB[1]{\lthtmlmathtype{#1}\egroup\lthtmlhboxmathA}%
\newcommand\lthtmlinlineZ{\egroup\expandafter\ifdim\dp\sizebox>0pt %
  \expandafter\centerinlinemath\fi\lthtmllogmath\lthtmlsetinline}
\newcommand\lthtmlinlinemathZ{\egroup\expandafter\ifdim\dp\sizebox>0pt %
  \expandafter\centerinlinemath\fi\lthtmllogmath\lthtmlsetmath}
\newcommand\lthtmlindisplaymathZ{\egroup %
  \centerinlinemath\lthtmllogmath\lthtmlsetmath}
\def\lthtmlsetinline{\hbox{\vrule width.1em \vtop{\vbox{%
  \kern.1em\copy\sizebox}\ifdim\dp\sizebox>0pt\kern.1em\else\kern.3pt\fi
  \ifdim\hsize>\wd\sizebox \hrule depth1pt\fi}}}
\def\lthtmlsetmath{\hbox{\vrule width.1em\kern-.05em\vtop{\vbox{%
  \kern.1em\kern0.8 pt\hbox{\hglue.17em\copy\sizebox\hglue0.8 pt}}\kern.3pt%
  \ifdim\dp\sizebox>0pt\kern.1em\fi \kern0.8 pt%
  \ifdim\hsize>\wd\sizebox \hrule depth1pt\fi}}}
\def\centerinlinemath{%
  \dimen1=\ifdim\ht\sizebox<\dp\sizebox \dp\sizebox\else\ht\sizebox\fi
  \advance\dimen1by.5pt \vrule width0pt height\dimen1 depth\dimen1 
 \dp\sizebox=\dimen1\ht\sizebox=\dimen1\relax}

\def\lthtmlcheckvsize{\ifdim\ht\sizebox<\vsize 
  \ifdim\wd\sizebox<\hsize\expandafter\hfill\fi \expandafter\vfill
  \else\expandafter\vss\fi}%
\providecommand{\selectlanguage}[1]{}%
\makeatletter \tracingstats = 1 


\begin{document}
\pagestyle{empty}\thispagestyle{empty}\lthtmltypeout{}%
\lthtmltypeout{latex2htmlLength hsize=\the\hsize}\lthtmltypeout{}%
\lthtmltypeout{latex2htmlLength vsize=\the\vsize}\lthtmltypeout{}%
\lthtmltypeout{latex2htmlLength hoffset=\the\hoffset}\lthtmltypeout{}%
\lthtmltypeout{latex2htmlLength voffset=\the\voffset}\lthtmltypeout{}%
\lthtmltypeout{latex2htmlLength topmargin=\the\topmargin}\lthtmltypeout{}%
\lthtmltypeout{latex2htmlLength topskip=\the\topskip}\lthtmltypeout{}%
\lthtmltypeout{latex2htmlLength headheight=\the\headheight}\lthtmltypeout{}%
\lthtmltypeout{latex2htmlLength headsep=\the\headsep}\lthtmltypeout{}%
\lthtmltypeout{latex2htmlLength parskip=\the\parskip}\lthtmltypeout{}%
\lthtmltypeout{latex2htmlLength oddsidemargin=\the\oddsidemargin}\lthtmltypeout{}%
\makeatletter
\if@twoside\lthtmltypeout{latex2htmlLength evensidemargin=\the\evensidemargin}%
\else\lthtmltypeout{latex2htmlLength evensidemargin=\the\oddsidemargin}\fi%
\lthtmltypeout{}%
\makeatother
\setcounter{page}{1}
\onecolumn

% !!! IMAGES START HERE !!!

{\newpage\clearpage
\lthtmlfigureA{capa14}%
\begin{capa}% \center
	\begin{center}
		\ABNTEXchapterfont\muctbf{\imprimirinstituicao} \\\muctbf{Pró-Reitoria Acadêmica – PROAC} \\\muctbf{\nomecurso}
\par
\vspace*{5cm}
\par
{\ABNTEXchapterfont\muctbf\imprimirautor}
\par
\vfill
				\ABNTEXchapterfont\bfseries\muctbf\imprimirtitulo
			\vfill
\par
\muctbf \imprimirlocal
\par
\muctbf \imprimirdata
	\end{center}
\end{capa}%
\lthtmlfigureZ
\lthtmlcheckvsize\clearpage}


\renewcommand{\folhaderostocontent}{
	\begin{center}
		{\ABNTEXchapterfont\muctbf\imprimirtitulo}
\par
\vspace*{2cm}
\par
\abntex@ifnotempty{\imprimirpreambulo}{%
			\hspace{.45\textwidth}
		\begin{minipage}{.5\textwidth}
			\SingleSpacing
			\imprimirpreambulo
		\end{minipage}%
		\vspace*{\fill}
		}%
\par
\vspace*{\fill}
\par
\muctbf{\imprimirlocal}
\par
\muctbf{\imprimirdata}
	\end{center}
}
{\newpage\clearpage
\lthtmlfigureA{fichacatalografica63}%
\begin{fichacatalografica}
% latex2html id marker 63

	\sffamily
		\vspace*{\fill} % Posição vertical
		\begin{center} % Minipage Centralizado
			\fbox{\begin{minipage}[c][8cm]{13.5cm} % Largura
				\small
				\imprimirautor
				%Sobrenome, Nome do autor
\par
\hspace{0.5cm} \imprimirtitulo  / \imprimirautor. --
				\imprimirlocal, \imprimirdata-
\par
\hspace{0.5cm} \thelastpage p. : il. (algumas color.) ; 30 cm.\\
\par
\hspace{0.5cm} \imprimirorientadorRotulo~\imprimirorientador\\
\par
\hspace{0.5cm}
				\parbox[t]{\textwidth}{\imprimirtipotrabalho~--~\imprimirinstituicao,
				\imprimirdata.}\\
\par
\hspace{0.5cm}
				1. Palavra-chave1.
				2. Palavra-chave2.
				2. Palavra-chave3.
				I. Orientador.
				II. Universidade xxx.
				III. Faculdade de xxx.
				IV. Título
			\end{minipage}}
		\end{center}
\end{fichacatalografica}%
\lthtmlfigureZ
\lthtmlcheckvsize\clearpage}

{\newpage\clearpage
\lthtmlfigureA{folhadeaprovacao85}%
\begin{folhadeaprovacao}
\par
\begin{center}
		{\ABNTEXchapterfont\large\imprimirautor}
\par
\vspace*{\fill}\vspace*{\fill}
		\begin{center}
			\ABNTEXchapterfont\bfseries\Large\imprimirtitulo
		\end{center}
		\vspace*{\fill}
\par
\hspace{.45\textwidth}
		\begin{minipage}{.5\textwidth}
			\imprimirpreambulo
		\end{minipage}%
		\vspace*{\fill}
	\end{center}
\par
Trabalho aprovado. \imprimirlocal, 24 de novembro de 2012:
\par
\assinatura{\textbf{\imprimirorientador} \\Orientador}
	\assinatura{\textbf{Professor} \\Convidado 1}
	\assinatura{\textbf{Professor} \\Convidado 2}
	%\assinatura{\textbf{Professor} \\Convidado 3}
	\par
\begin{center}
		\vspace*{0.5cm}
		{\large\imprimirlocal}
		\par
{\large\imprimirdata}
		\vspace*{1cm}
	\end{center}
\par
\end{folhadeaprovacao}%
\lthtmlfigureZ
\lthtmlcheckvsize\clearpage}

{\newpage\clearpage
\lthtmlfigureA{dedicatoria142}%
\begin{dedicatoria}
	\vspace*{\fill}
		\centering
		\noindent
		\textit{ 
			Dedico este trabalho ao meu professor orientador MSc. \imprimirorientador por me acolher e aconselhar passando seus conhecimentos sem pestanejos e me aceitando como último orientando antes de sua jornada em busca do título de doutorado. Dedico também ao professor MSc. Hugo Vieira Lucena de Souza por todos seus conselhos, ensinamentos e correções que foram possível para conclusão deste título. Também dedico este trabalho e ao professor DSc. Erick France Meira de Souza por sua amizade, suporte e preciosos conselhos dados.
		} \vspace*{\fill}
\end{dedicatoria}%
\lthtmlfigureZ
\lthtmlcheckvsize\clearpage}

{\newpage\clearpage
\lthtmlfigureA{agradecimentos151}%
\begin{agradecimentos}
	Gostaria de agradecer minha avó Carminha (in memorian) e primo Luiz Felipe Abreu (in memorian).
\par
Gostaria de agradecer aos meu pais Mathura Pati Devi Das e Subala Das por sempre prezar e batalhar pela educação dos seu filhos, sem medir esforços para que todos pudesse se graduar e obter a formação acadêmica necessária.
\par
Meus agradecimentos também a minhas irmã Jahnavi Caran por me apoiar, dar suporte e aconselhar na minha trajetória acadêmica. Meus sinceros agradecimentos e minha eterna gratidão, irmã.
\par
A todos os professores e pessoas não citadas diretamente neste texto, mas que participaram e contribuíram de alguma forma com a minha formação, meu muito obrigado.
\par
\end{agradecimentos}%
\lthtmlfigureZ
\lthtmlcheckvsize\clearpage}



\setlength{\absparsep}{18pt}%

\setlength{\absparsep}{18pt}
{\newpage\clearpage
\lthtmlfigureA{resumo165}%
\begin{resumo}
	%
Política de transparência públicas e fiscalização são essenciais para o controle do cidadão no controle e fiscalização de gastos públicos. Porém, tais cidadão certamente utilizaram como canal mediador entre entidade portadora dos dados para fiscalização pública e o cidadão que atua como fiscal neste cenário. Excluindo assim, pessoas leigas em elementos com teor tecnológico.
\par
porém essas ferramentas apesar de apresentarem soluções eficazes muitas vezes não possuem um processo de inclusão realmente eficiente e condizente com a realidade do usuário final. 
\par
Muitas dar aplicações existentes direcionadas ao público de portadores do espectro autista possuem como proposito auxiliar na comunicação e na aprendizagem da língua portuguesa ou de atividades do cotidiano, satisfazendo apenas campo específicos e havendo a real necessidade de mecanismos que auxiliem em outras áreas.
\par
Portanto o presente trabalho tem o intuito de gerar uma aplicação de apoio ao ensino dos numerais de 1 a 9 para crianças portadoras do transtorno do espectro autista, utilizando um processo criativo baseado no usuário
\par
, o Design Thinking, cujo objetivo principal é a resolução de problemas com foco primordial no ser humano, levando em consideração a realidade do público alvo, satisfazendo as suas necessidades e respeitando as suas limitações.  
\par
Através de ciclo de etapas baseados nesta metodologia foi possível realizar o desenvolvimento da solução proposta, cuja última etapa consiste na validação da solução, através da aplicação da mesma em campo por crianças portadoras do espectro.
	\textbf{Palavras-chave}: latex. abntex. editoração de texto.
\end{resumo}%
\lthtmlfigureZ
\lthtmlcheckvsize\clearpage}

{\newpage\clearpage
\lthtmlfigureA{resumo211}%
\begin{resumo}[Abstract]
	\begin{otherlanguage*}{english}
		This is the english abstract.
\par
\vspace{\onelineskip}
\par
\noindent 
		\textbf{Keywords}: latex. abntex. text editoration.
	\end{otherlanguage*}
\end{resumo}%
\lthtmlfigureZ
\lthtmlcheckvsize\clearpage}

{\newpage\clearpage
\lthtmlfigureA{siglas232}%
\begin{siglas}
\item[API] Application Programming Interface
\item[LUIS] Language Understanding Intelligent Service
\item[JSON] JavaScript Object Notation
\item[IA] Inteligência Artificial
\item[PLN] Processamento de Linguagem Natural 
\item[REST ] Representational State Transfer
\end{siglas}%
\lthtmlfigureZ
\lthtmlcheckvsize\clearpage}

\stepcounter{chapter}
\stepcounter{section}
\stepcounter{section}
\stepcounter{subsection}
\stepcounter{subsection}
\stepcounter{section}
\stepcounter{section}
\stepcounter{chapter}
\stepcounter{section}
\stepcounter{section}
\stepcounter{subsection}
\stepcounter{section}


\setlength{\labelsep}{0pt}%

\setlength{\labelsep}{0pt}

\end{document}
