%Abstract%
\section*{\centering{ABSTRACT}} 
\singlespacing 
\noindent 
Facial appearance may change as age progresses, aging the human face for facial recognition is still a challenging problem, and one of the challenges faced in it is to achieve temporal invariance.
This work proposes a technological solution that is believed to contribute to the solution of this problem, will be used for the proposal the celebrity data set called CACD (Cross-Age Celebrity Dataset), the dataset contains 163,446 images of 2,000 celebrities collected from the Internet, where it is possible to visualize images of celebrities along their careers, being they male or female, in the age range between 16 and 62 years. Images are collected from search engines using celebrity name and year (2004-2013) as keywords. In this dataset were manually removed noisy images of celebrities with a rating of less than or equal to five. This image bank was designed primarily for cross-age face recognition and recovery. The choice for the dataset was based on it being by far the largest dataset of age-matched faces publicly available. This work will present the application of Siamese Neural Networks to treat the comparative characteristics extracted from these images, giving the result of the greater proximity between them.
A binary tree structure is used for dimensionality reduction. A significant improvement in accuracy can be observed in this method.
\\
\\
{\bf Key-words:} $<$Facial recognition. Facial Aging. Celebrities. Neural networks.
$>$
